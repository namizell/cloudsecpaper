\documentclass[11pt]{article} 

\usepackage{bookmark}
\usepackage[margin=1in]{geometry}
\usepackage{multicol}

\title{Creating A Crowd Sourced Spam Metric Repository With AWS}
\author{Adam Agee \and Jamal Cooper \and Noah Mizell \and James Odom \and William Setzer \\ \{acagee, cooperjd, namizell, odom021, wjsetzer\}@uab.edu}

\begin{document}

\maketitle

% ABSTRACT

\begin{multicols}{2}

  % unnumbered section
  \section*{Abstract}

  Talk about the whole project briefly in this section. Summarize the project. This should not be more than five or six sentences long.

  \section{Introduction} 

Talk about the problem, why it is important, why it is difficult to solve, what have others done to solve it. Why are those solutions not useful. What is your proposed approach. Why is your approach going to be better than others?
 At the end of the intro, mention how the rest of the paper is organized (e.g., “The rest of the paper is organized as follows: Section 2 
provides background information. We introduce the problem motivation in Section 3 ….”)

  \section{Motivation}
 Talk about the motivation of the paper. This should be a good expansion of the Need part of your NABC document.
 
  \section{Background}
Do we need to know some background information about the things used in the project? Don’t make this section too long with trivial information though.

  \section{Technical Approach}
What is your technical approach? What are you doing to solve the problem? What is the high level design and components of your system? Give a block diagram of the system. Describe the components. Show how the data flows in your system using a diagram. Talk about implementation. Provide screenshots of your project app or software.

  \section{Results}

  \section{Discussion}
 How does your work compare with existing related work? Describe the related work in the area and their limitations. If possible, do some experiments to show either comparison with related systems, or provide a discussion of the benefits of your system over others. (A table with projects in rows and various features in columns is a nice way to show what others lack and what you can provide). What are the limitations of your work? What can be good future work? What are the lessons learned?
 
  \section{Task Distribution}
 How did you divide the project among the group members?

  \section{Conclusion}
Conclude the paper by looking back and commenting on what you did and about the problem and your solution.

  \section*{Acknowledgements}
Acknowledge the help you received from various people.
 
Gary Warner
 

  Neil Schwartzman

  \section*{References}
  E. Blanzieri and A. Bryl. A survey of learning-based techniques of email spam filtering. In Artificial Intelligence Review, Vol. 29, Issue 1, pp 63-92, March 2008.
K. Thomas, C. Grier, J. Ma, V. Paxson, and D. Song. Design and Evaluation of a Real-Time URL Spam Filtering Service. IEEE. July 2011.
M. Aun, B. Goi, and V. Kim. Cloud Enabled Spam Filtering Services: Challenges and Opportunities. IEEE Conference on Sustainable Utilization and Development in Engineering and Technology. October 2011.

\end{multicols} 
\end{document}

